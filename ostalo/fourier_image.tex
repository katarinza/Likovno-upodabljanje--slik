\chapter{Fourierova vrsta in transformacija}
%%% Najprej navedemo motivacijo, zakaj bomo potrebovali Four. vrste in njeno transformacijo.
\section{Motivacija}
Povemo, da bomo naredili Foourierovo transformaijo, ki nam prenesla funkcijo iz �asovne v frekven�no domeno. S tem bomo nalizirali slike in deali Fourierove filtracijo. Naredi� �e primer, ki poka�e frekven�no in �asovno domeno. Verjetno kak�no nihanje itd. nekaj pa� ...
%%% Predstavitev Fourierove vrste, skupaj z izreki, dokazi, lastnostmi, ...
\section{Fourierova vrsta}
%%%
\section{Fourierova transformacija}
%%%
\section{Algoritem za FFT in inverzno FFT}
Naredimo tudi primer. Zelo znano je nekaj z obra�anjem bitov.

\section{Uporaba Fourierove transformacije in filtrov}
Filtriranje slike $f$ s filtrom $G$ zahteva naslednje korake:
\begin{enumerate}
\item Predprocesiranje dane slike (npr.\ histogram equalization, ali preslikava slike v ve�jo $2n\times 2^n$ tabelo na kateri uporabimo FFT, �e je to potrebno).
\item Sliko $f$ s Fourierovo transformacijo preslikamo v sliko kompleksno $F$ v prostoru s frekven�no domeno.
\item Matriko $F$ po elementih mno�imo s simetri�nim kompleksnim filtrom $G$ ($G$ dobimo s FT filtra jedra $\textup{ker} g$).
\item Transformacija $F\circ G$ nazaj v spatial domeno, v relano $f*g$ z uporabo inverzne FFT.
\item Naredimo �e postprocessing, �e je potrebno (npr.\ �e smo v predprocesiranju sliko pove�ali).
\end{enumerate}