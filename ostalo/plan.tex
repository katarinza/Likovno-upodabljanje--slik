\chapter{Likovna umetnost in matematika skozi zgodovino}
%
Zgodovinski pregled dogajanja na področju matematike in likovne umetnosti.

Vprašanja:
\begin{itemize}
  \item Kaj vključiti noter?
  \item Morda tko res samo pregled dogajanja, brez podrobnosti ... da se dobi občutek.
  \item Zaključimo pri tem, ko se začne računalnik uporabljati ...
  \item ...
\end{itemize}

%% PRIMERI GENERATIVNE UMETNOSTI
\chapter{Primeri generativne umetnosti}
%
\section{Random Art}
\section{Bogdan Soban}
\section{Teja Krašek}
\section{Scott Draves}
\section{TSP}
% Mona Lisa TSP
% http://www.tsp.gatech.edu/data/art/index.html
\begin{figure}[htb]
\includegraphics[width=0.4\textwidth]{monalisacurve}
\end{figure}
\section{Ostali strici in tete vredni omembe}

%% OPIS SLIKE
\chapter{Opis slike}

\begin{enumerate}
  \item Slika je opisana s parametri:
    \begin{itemize}
      \item Sliko narišemo z neko fiksno funkcijo $f(p_1, ..., p_n)$, ki smo jo napogramirali
      \item različni izbori parametrov $p_1, ..., p_n$ dajo različne slike
      \item (Ali je electricsheep oz. Flame algorithm od Scott Draves-a te vrste?)
      \item primeri: IFS fraktali, electricsheep?, ...
    \end{itemize}
    
  [MATEMATIKA: IFS fraktali, geometriska konstrukcija kake lepe slikice]

  \item Slika je opisana kot funkcija koordinate točke $(x,y)$:
    \begin{itemize}
      \item Naivno z drevesi
      \item Kako to naredi Random art (aciklični grafi, oz. računska vezja)
    \end{itemize}
    
   [MATEMATIKA: drevesa, aciklični grafi/računska vezja]

  \item Slika je opisana z grafičnimi elementi (spline, poligoni, rasterji, ...) in grafičnimi operacijami (blur, sharpen, edge detect, color transform, ...)

   [RAČUNALNIĹ KA GRAFIKA: kaj so tipični grafični elementi, ki jih imamo v grafičnih knjižnicah in grafičnih karticah]

\item Kolaž: sliko sestavimo iz obstoječih elementov (kosov drugih slik, barve poberemo iz fotografije) ali pa vzamemo obstoječo sliko za "vodilo": fotografijo aproksimiramo z majhnim številom barvnih trikotnikov
\end{enumerate}

\chapter{Nadziranje in odkrivanje vizualnih elementov}

\textbf{Diskusija:}
\begin{enumerate}
  \item kaj je lepa slika? (Marko Grobelnik je imel študenta, ki je analiziral, katere slike v Random Art se ljudem zdijo "lepe".)
  \item Če naredimo Fourierovo transformacijo slike in se igramo z njo, kaj se zgodi?
\end{enumerate}

\noindent \textbf{Kaj delamo?}
\begin{enumerate}
  \item Kako določamo barvno shemo, kako pazimo, da so lepe barve? \\
   Kako izmerimo, ali je barvna shema primerna? (Histogram barv)

   [MATEMATIKA: barvni modeli RGB, HSV, ...]

  \item Kako določamo katere grafične elemente bomo uporabljali?

  \item Purpose of Image processing \\
   The purpose of image processing is divided into 5 groups. They are:
    \begin{enumerate}
      \item Visualization - Observe the objects that are not visible.
      \item Image sharpening and restoration - To create a better image.
      \item Image retrieval - Seek for the image of interest.
      \item Measurement of pattern  Measures various objects in an image.
      \item Image Recognition Distinguish the objects in an image.
    \end{enumerate}
  \item Kako pazimo, da ni preostrih prehodov, oz. da so ravno v pravi meri?\\
   Kako izmerimo, ali so prehodi ustrezni? (Fourier, detekcija robov)

   [MATEMATIKA: polja, gradienti, detekcija robov]

  \item Kako merimo/nadziramo količino "podrobnosti"?

  \item Kako nadziramo/merimo količini simetričnosti in razporeditev v sliki?

  \item Kako nadziramo fokusne točke in osnovne delitve slike na območja (horizont, pod njim zemlja nad njim nebo).

  \item Interaktivnost: slika, ki je odvisna od vhoda v realnem času (kamera, zvok).\\
   Kako dovolj hitro risati sliko?
\end{enumerate}

\chapter{Implementacija}